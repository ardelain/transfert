\documentclass[5pt]{article}
\usepackage{graphicx}
\newcommand\tab[1][1cm]{\hspace*{#1}}
\renewcommand{\abstractname}{Architecture des système}
\usepackage{array, tabularx}
\newcolumntype{Y}{>{\raggedright\arraybackslash}X}

\usepackage{geometry}
\geometry{hmargin=1cm,vmargin=1cm}

\makeatletter
\renewcommand{\subsubsection}{\@startsection {section}{1}{\z@}%
             {-1ex \@plus -1ex \@minus -.2ex}%
             {1ex \@plus.2ex}%
             {\normalfont\scriptsize\sffamily\bfseries}}
\renewcommand{\subsection}{\@startsection {section}{1}{\z@}%
             {-1ex \@plus -1ex \@minus -.1ex}%
             {1ex \@plus.1ex}%
             {\normalfont\small\sffamily\bfseries}}
\makeatother

\begin{document}
\begin{scriptsize}
\title{Assembleur}
\date{}
\begin{abstract}
Fiche Programmation assembleur x86 
\end{abstract}
\subsection{Introduction}
Assembleur x86  = langage de génération 1\\
Avantage : exécution très rapide \\
Inconvénients :
très bas niveau (très peu user-friendly); pas portable; beaucoup de contraintes de programmation (langage très technique)\\
Un assembleur par architecture de processeur : famille x86 \\
– famille de processeurs incluant Intel Pentium, Intel i7, AMD Phenom, AMD Athlon, etc. –ex processeur 8086 : 1978 \\
manipule 8 bits, 16 bits ou 32 bits ;
syntaxe Intel (et non pas syntaxe ATetT) ;
environnement : système d'exploitation Linux, logiciel nasm\\
Exectution Exemple:\\
\begin{tabularx}{\linewidth}{|Y|Y|Y|Y|}
\hline
nasm -f elf hello-world.asm &
ld -m elf\_i386 -o hello-world hello-world.o
./hello-world &
Désassemblage : objdump -d hello-world -M intel&
Déboguage : gdb hello-world 
\\ \hline
\end{tabularx}

\subsection{Registres et mémoire }
\begin{tabularx}{\linewidth}{|X|}
\hline
Registres : variables internes au processeur , accès très rapide, mais faible nombre de registres , les registres sont spécialisés\\
Registres de données :\\
registres 32 bits : EAX, EBX, ECX, EDX  : très utilisés (manipulation des données, appel de fonctions, etc.)\\ chacun comprend un registre 16 bits (AX, BX, CX, DX)\\ chaque registre 16 bits comprend deux registres 8 bits (AH, AL, BH, BL, CH, CL, DH, DL)
\\\hline
Mémoire : décomposée en segments (blocs) et en offsets (index) \\
variables globales, variables locales, allocation dynamique \\
pile d'appels de fonction (schéma)\\
Registres de segments : code : CS ; données : DS ; supplémentaires : ES, FS, GS ; pile : SS
\\\hline
Registre d'index : code : IP ; données : DI, SI; pile : SP, BP\\
Registre des indicateurs : indicateur de zéro : ZF ; indicateur de signe : SF ; indicateur d'interruptions : IF ; etc\\
\\ \hline
\end{tabularx}
Adressage :\\
\begin{tabularx}{\linewidth}{|Y|Y|}
\hline
adressage direct : le contenu de la case 0x100 se dit [0x100] (ou DS:[0x100]) &
adressage indirect : le contenu de la case dont l'adresse est la valeur de DI (ou de SI) : [DI] (très utilisé)\\\hline
adressage basé : le contenu de la case dont l'adresse est la valeur de BX : [BX] (ou avec BP, mais utilise le segment SS dans ce cas) (peu utilisé)&
adressage indirect basé avec déplacement : [DI+BX+4] (très peu utilisé
\\ \hline
\end{tabularx}
\subsection{Instructions assembleur }
\begin{tabularx}{0.5\linewidth}{|Y|}
\hline
•Affectation : MOV destination, valeur \\
Remarques : 'valeur' peut être une constante, un registre, une zone mémoire et une zone mémoire peut être désignée par : [constante], [registre], [registre+constante], [registre+BP+constante] 
\\on peut préciser le segment en utilisant segment: [offset], ou [segment:offset]\\
\\ \hline
\end{tabularx}
\begin{tabularx}{0.5\linewidth}{|Y|}
\hline
•Contraintes \\
on ne peut pas désigner deux zones mémoire dans un MOV, il faut passer par un registre intermédiaire \\
on ne peut pas modifier avec un MOV certains registres (IP, registres de segments) \\
implicitement, DS est utilisé, sauf avec B
\\ \hline
\end{tabularx}
\textbf{Instructions}  \\
\noindent
\begin{tabularx}{\linewidth}{|>{\setlength\hsize{0.5\hsize}}Y|>{\setlength\hsize{0.8\hsize}}Y|>{\setlength\hsize{1.5\hsize}}Y|>{\setlength\hsize{1.2\hsize}}Y|}
\hline
Type & Nom & Remarque & contrainte\\ \hline
Logiques  & AND destination, valeur&&\\ \hline
& OR destination, valeur&&\\ \hline
& NOT destination&&\\ \hline
& XOR destination, valeur&&\\ \hline
arithmétiques   & ADD destination, valeur&&\\ \hline
& SUB destination, valeur&&\\ \hline
& IMUL registre1, opérande2& multiplie r1 et o2, et stocke résultat dans r1&\\ \hline
& IDIV opérande &divise le contenu de EDX:EAX par l'opérande, place le quotient dans EAX et le reste dans EDX&\\ \hline
Comparaisons& CMP a, b&&\\ \hline
Branchements   & JMP destination  &(inconditionnels)&\\ \hline
Branchements   & JE/JNE/JG/JGE/JL/JLE dest &(conditionnels)si égalité ZF=0etCF=0/si non-égalité ZF=0/si supérieur ZF=0etSF=OF/si sup et eg/ si inf /si inférieur ou égal ZF=1ouSF!=OF&\\ \hline
Remarques/autres& JMP segment:[offset]&&\\ \hline
Boucle & LOOP label &décrémente CX et effectue un saut au label désigné&\\ \hline
Fonctions & call label&pour appeler la fonction &\\ \hline
Fonctions & ret& pour sortir de la fonction&\\ \hline
Pile &&utilisée pour les arguments des fonctions, et pour sauvegarder le contexte lors d'interruption. utilise SS:ESP&\\ \hline
Pile & push valeur &empilement (ESP diminue)&\\ \hline
Pile & pop valeur &dépilement (ESP augmente)&\\ \hline
Interruption & int numéro&appeler une interruption&\\ \hline
Interruption & iret&sortir d'une interruption&\\ \hline
\end{tabularx}
\\
Les registres d’offset :esi,edi,ebp et esp.\\

\noindent
\begin{tabularx}{0.5\linewidth}{|>{\setlength\hsize{0.5\hsize}}Y|>{\setlength\hsize{1.5\hsize}}Y|}
\hline
Nom & Utilisation\\ \hline
0x0 à 0x7 & processeur \\ \hline
0x8 à 0xf & périphériques \\ \hline
0x10 & vidéo \\ \hline
0x13  & accès aux disques\\ \hline
0x16  & clavier \\ \hline
\end{tabularx}
\begin{tabularx}{0.5\linewidth}{|>{\setlength\hsize{0.5\hsize}}Y|>{\setlength\hsize{1.5\hsize}}Y|}
\hline
Nom & Utilisation\\ \hline
0x1C  & horloge \\ \hline
0x20  &  DOS : terminer un programme  \\ \hline
0x21  & DOS : API \\ \hline
0x28  & DOS : boucle d'attente du shel \\ \hline
80h   & DOS : Unix : API, util pour les appels systèmes Unix   \\ \hline
\end{tabularx}
\begin{tabularx}{0.5\linewidth}{|X|}
\hline
Appels systèmes sous Linux \\
$\tab$unistd.h contient les numéros de chaque appel système \\
$\tab$le numéro de l'appel système est dans EAX \\
$\tab$les paramètres sont passés via EBX, ECX, EDX, ESI, EDI, EBP (dans l'ordre)
\\\hline
\end{tabularx}
\subsection{Exemples}
\textbf{... exemple calcule, boucle, algos,...}  \\
\noindent
\begin{tabularx}{0.333\linewidth}{|Y|}
\hline
$\tab$//xor ax,ax\\i=0
$\tab$mov ax,0\\i=0
$\tab$mov bx,100\\j=100
debut:\\
$\tab$cmp ax,20\\
$\tab$jl fin\\
$\tab$cmp bx,50\\
$\tab$jge fin\\
$\tab$mov dx,ax\\
$\tab$imul dx,2\\
$\tab$sub bx,dx\\
$\tab$cmp bx,10\\
$\tab$jge else\\
$\tab$add bx,120\\
$\tab$jmp endif\\
$\tab$else:...\\
$\tab$endif:\\
$\tab$inc ax\\
$\tab$jmp debut\\
fin:\\
OU Inverse\\
debut :\\
$\tab$. . .\\
$\tab$inc   ax;\\
$\tab$cmp   ax,10\\
$\tab$jg debut
\\ \hline
\end{tabularx}
\begin{tabularx}{0.333\linewidth}{|Y|}
\hline
section .data\\
table: db 'Hello', 3\\
size: equ \$ - table\\
char: db 0\\
section .text\\
global \_start\\
\_start:\\
    $\tab$mov al, [table]; on instancie le minimum avec le premier élément de la table\\
    $\tab$mov ebx, 0 ;offset\\
find\_minimum:\\
.loop:\\
    $\tab$cmp ebx, size\\
    $\tab$jae subtract\\
    $\tab$cmp al, [table+ebx]\\
    $\tab$jb .no\_update ; Si le caractère n'est pas < minimum on ne fait rien\\
    $\tab$mov al, [table+ebx]\\
    $\tab$.no\_update:\\
    $\tab$inc ebx\\
    $\tab$jmp .loop\\
subtract:\\
    $\tab$mov ebx, 0\\
.loop:\\
    $\tab$cmp ebx, size\\
    $\tab$jae end\_loop\\
    $\tab$sub [table+ebx], al ; on soustrait la valeur\\
    $\tab$; et on affiche pour vérifier que tout fonctionne\\
    $\tab$push eax\\
    $\tab$mov dl, [table+ebx]\\
    $\tab$push ebx\\
    $\tab$mov eax, 4\\
    $\tab$mov ebx, 1\\
    $\tab$mov [char], dl\\
    $\tab$mov ecx, char\\
    $\tab$mov edx, 1\\
    $\tab$int 0x80 \\
    $\tab$pop ebx\\
    $\tab$pop eax \\
    $\tab$inc ebx\\
    $\tab$jmp .loop\\
end\_loop:\\
    $\tab$mov eax, 1\\
    $\tab$mov ebx, 0\\
    $\tab$int 0x80\\
\\ \hline
\end{tabularx}
\begin{tabularx}{0.333\linewidth}{|Y|}
\hline
tri par bulle, pour un tableau table de taille n\\
section .data\\
table: db 'teststring'\\
n: equ \$ - table\\
section .text\\
global \_start\\
\_start:\\
mov al, 0\\
.loop\_i: ; i est implémenté avec al\\
    $\tab$cmp al, n\\
    $\tab$jae .end\_loop\_i\\
    $\tab$mov ebx, 1 ; j est stocké dans ebx (taille 32 bits pour les offsets)\\
    $\tab$.loop\_j:\\
    $\tab$mov ecx, n\\
    $\tab$sub ecx, eax\\
    $\tab$cmp ebx, ecx ; j n'a pas besoin d'aller au dela de n-i-1\\
    $\tab$jae .end\_loop\_j\\
        $\tab\tab$;on place [table + ebx] dans cl pour pouvoir utiliser cmp\\
        $\tab\tab$mov cl, [table + ebx]\\
        $\tab\tab$cmp cl, [table + ebx - 1]\\
        $\tab\tab$jae .no\_swap ; si t[j] > t[j-1] il n'y a rien à faire\\
            $\tab$ $\tab$; on met t[j-1] dans t[j]\\
           	$\tab$ $\tab$mov dl, [table + ebx-1]\\
            $\tab$ $\tab$mov [table + ebx], dl\\
            $\tab$ $\tab$mov [table + ebx - 1], cl ; on met t[j] dans t[j-1]\\
        $\tab\tab$.no\_swap:\\
        $\tab\tab$inc ebx\\
    $\tab$jmp .loop\_j\\
    $\tab$.end\_loop\_j:\\
    $\tab$inc al\\
jmp .loop\_i\\
.end\_loop\_i:\\
 mov eax, 0\\
.loop\_print:\\
    $\tab$cmp eax, n\\
    $\tab$jae end\_print\\
    $\tab$mov ecx, table\\
    $\tab$add ecx, eax\\
    $\tab$push eax\\
    $\tab$mov eax, 4\\
    $\tab$ mov ebx, 1\\
    $\tab$mov edx, 1\\
    $\tab$int 0x80\\
    $\tab$pop eax\\
    $\tab$inc eax\\
    $\tab$jmp .loop\_print\\
end\_print:\\
    $\tab$mov eax, 1\\
    $\tab$mov ebx, 0\\
    $\tab$int 0x80\\
\\ \hline
\end{tabularx}
\\
...
\\
\\
\begin{tabularx}{0.333\linewidth}{|Y|}
\hline
accès à la libc en assembleur:\\
utilisation de "extern" et utilisation de "main" au lieu de "\_start" (car "\_start" est redéfini par le linker) : la fonction doit respecter l'interface imposée (pour ne pas avoir d'effet de bord)\\
(exemple avec extern puts, exit)\\
\hline
int main() \{\\
$\tab$int value = 100;\\
$\tab$asm( "mov eax, \%0;"\\
$\tab\tab$"inc eax;"\\
$\tab\tab$"mov \%0, eax;"\\
$\tab\tab$:"=r"(value):"r"(value)\\
$\tab$);\\
$\tab$printf("Hello world");\\
$\tab$printf("value is \%d (should not be ", value);\\
$\tab$printf("equal to 100 anymore)");\\
$\tab$return 0; \\
\}\\
\\ \hline
\end{tabularx}
\begin{tabularx}{0.333\linewidth}{|Y|}
\hline
\\ \hline
\end{tabularx}
\begin{tabularx}{0.333\linewidth}{|Y|}
\hline
\\ \hline
\end{tabularx}
\\
\begin{tabular}{|c|c|}
\hline
\begin{tabularx}{0.5\linewidth}{|Y|}
\hline
compiler avec "gcc -masm=intel -o methode2 methode2.c"\\
Pour accéder aux variables en écriture, on utilise "=r" (pour registre)\\
Pour accéder aux variables en lecture, on utilise "r"\\
 Dans la partie assembleur, on utilise \%0, \%1, etc
\\ \hline
\end{tabularx} 
&
\begin{tabularx}{0.5\linewidth}{|Y|}
\hline
optimiser le code assembleur produit par du C  \\
gcc -S -masm=intel methode3.c  \\
modifier le fichier methode3.s  \\
gcc -o methode3 methode3.s
\\ \hline
\end{tabularx} 
\\ \hline
\end{tabular}
\subsection{Autres}
•
$\bullet\tab$
\end{scriptsize}
\end{document}