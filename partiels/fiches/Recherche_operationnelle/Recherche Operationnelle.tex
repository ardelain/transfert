\documentclass[5pt]{article}
\usepackage{graphicx}
\newcommand\tab[1][1cm]{\hspace*{#1}}
\renewcommand{\abstractname}{Modele Lineaire de Recherche Operationnelle}
\usepackage{array, tabularx}
\newcolumntype{Y}{>{\raggedright\arraybackslash}X}

\usepackage{geometry}
\geometry{hmargin=1cm,vmargin=1cm}

\begin{document}
\begin{scriptsize}
\title{Recherche Operationnelle}
\date{}
\begin{abstract}
Fiche
\end{abstract}
\subsection{Formulations }
Intro:\\
Model mathematique\\
Recherche d'operations\\
Programmation Lineaire (LP): minimiser ou maximiser une fonction lieaire et satisfaisant des egalitiées et/ou des inégalitées de contraintes ou de restrictions\\
Application...\\
\\
Les fonctions lineaires:\\
Les probleme de programmation lineaire:\\
Formulation\\
Forme standard et convertion\\
\\
Terminologie:\\
\\
Solution Geometrique/Graphique \\
\\
probleme LP:\\
-infaisable\\
-sans borne\\
\subsection{Methode du Simplex }
Extension\\
Restriction\\
variable basique et non basique\\
dictionnaire\\
solutions basique\\
meilleurs solution faisable:\\
-nouveux dictionnaire\\
-iteration du simplex\\
-variable d'entree et choix\\
-variable sortie / recherche\\
-Pivot\\
-second iteration du simplex\\
-troisieme\\
-quatrieme\\
-Critere d'arret\\
-solution optimal multiple\\
Exemple\\
Exemple\\
$\bullet\tab$
\subsection{Methode du Simplex a deux phases}
probleme des cycles\\
Choix\\
Methode Lexicographique\\
Probleme LP auxiliaire\\
Methode simplexe a deux phases\\
Theoreme fondamental de programtion lineaire\\
Exemple\\
Probleme LP artificiel\\
Autre Methode simplexe a deux phases\\
Methode Simplex Big-M\\
Probleme Klee-Minty\\
Algorithme a temps polynomiale\\
\subsection{Theorie de la dualité}
Utilité\\
Limites superieurs\\
problems Primal et Dual 
\subsection{Methode du Simplex revisité}
Utilité\\
Matrice\\
Methode\\
Exemple\\
\subsection{Analyse sensitive}
Utilité\\
Methode\\
Exemple\\
\textbf{tab titre}  \\
\noindent
\begin{tabularx}{\linewidth}{|Y|Y|Y|}
\hline
... & ... & ...\\ \hline
\\ \hline
\end{tabularx} 
\end{scriptsize}
\end{document}