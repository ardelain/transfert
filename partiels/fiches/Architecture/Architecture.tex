\documentclass[5pt]{article}
\usepackage{graphicx}
\newcommand\tab[1][1cm]{\hspace*{#1}}

\renewcommand{\abstractname}{Architecture des système}
\usepackage{array, tabularx}
\newcolumntype{Y}{>{\raggedright\arraybackslash}X}

\usepackage{geometry}
\geometry{hmargin=1cm,vmargin=1cm}

\makeatletter
\renewcommand{\subsubsection}{\@startsection {section}{1}{\z@}%
             {-1ex \@plus -1ex \@minus -.2ex}%
             {1ex \@plus.2ex}%
             {\normalfont\scriptsize\sffamily\bfseries}}
\renewcommand{\subsection}{\@startsection {section}{1}{\z@}%
             {-1ex \@plus -1ex \@minus -.1ex}%
             {1ex \@plus.1ex}%
             {\normalfont\small\sffamily\bfseries}}
\makeatother

\begin{document}
\begin{scriptsize}
\title{Architecture des système}
\date{}
\begin{abstract}
Fiche
\end{abstract}
\subsection{Introduction}
\textbf{tab titre}  \\
\noindent
\begin{tabularx}{\linewidth}{|Y|Y|Y|}
\hline
... & ... & ...\\ \hline
\\ \hline
\end{tabularx} 
\subsection{Fonctionnement d'un ordinateur}
•stockage$\tab$
•traitement\\
$\circ$ par des circuits simples (sans notion de temps et sans stockage)
$\circ$ par des circuits complexes (avec notions de temps et de stockage)\
$\circ$ entre circuits complexes
\subsubsection{Représentation des valeurs}
Bit : unité de stockage de l'information : 0 ou 1\\
Stockage de valeurs plus complexes : Comment sont stockées ces valeurs?(entiers positifs, entiers naturels, nombres réels, caractères, chaînes de caractère)\\
\begin{tabular}{|c|c|}
\hline
\begin{tabularx}{0.5\linewidth}{|Y|}
\hline
\textbf{• 1 Entiers positifs}
\\\hline
on utilise la base en indice\\
•Comment connaître la valeur décimale d'un nombre écrit dans une base b ?\\
-soit un nombre binaire (an-1,an-2,...,a2,a1,a0), chaque ai étant écrit dans la base b\\
-la valeur de (an-1,an-2,...,a2,a1,a0) est sum(i=0..n-1)ai.$b^{i}$
\\\hline
•Comment connaître l'écriture dans une base b d'un nombre n écrit en base décimale ?\\
-écrire les puissances de b\\
-calculer c et k tel que : – c est compris entre 0 et b-1 – c.$b^{k}$ $\leq$ n $<$ (c+1).$b^{k}$\\
-retirer du nombre actuel c.$b^{k}$ : c est le k-ème chiffre de l'écriture de n dans la base b\\
-recommencer jusqu'à obtenir 0
\\\hline
• Comment passer d'une base $b_{1}$ à une base $b_{2}$ ?\\
- passer de la base $b_{1}$ à la base 10
- passer de la base 10 à la base $b_{2}$\\
• Comment passer d'une base 16 à une base 2 rapidement ?\\
- traduire chaque chiffre hexadécimal en 4 bits\\
• Comment passer d'une base 2 à une base 16 rapidement ?\\
- lire les bits par blocs de 4, en commençant à droite\\
- traduire chaque bloc de 4 bits indépendemment (car 16=$2^{4}$)
\\ \hline
Les Additions :\\
comme sur les décimaux ; attention aux retenues ;\\
attention aux débordements de capacité (le résultat peut avoir plus de chiffres que les opérandes)\\
Ex : faire l'addition de 39 et de 46 en base 2 \\
39 = 1001112 et 46 = 1011102 \\
on obtient bien 10101012 = 85
\\ \hline
\end{tabularx} 
\begin{tabularx}{0.5\linewidth}{|Y|}
\hline
\textbf{• 2 Entiers naturels}\\\hline
Pb : il faut un moyen pour stocker les nombres\\ négatifs \\
Solution initiale : un bit de signe (le plus à gauche)\\
Ex : représenter 39 et -39 en base 2, avec un bit de signe, sur 8 bits\\
39 = 100111 ;
39 sur 8 bits : (0)0100111 ;
-39 sur 8 bits : (1)0100111
\\\hline
Pb : les additions et les soustractions sont à gérer différemment \\
0 et -0 ont deux représentations différentes \\
Solution intermédiaire : complément logique (ou complément à 1) \\
pour les entiers négatifs, tous les bits sont inversés \\
Ex: représenter 39 et -39 en base 2, en complément logique, sur 8 bits \\
39 = 100111 ;
39 sur 8 bits : 00100111 ;
-39 sur 8 bits : 11011000
\\\hline
Remarque : les nombres positifs commencent par 0, les nombres négatifs par 1 \\
Comment ajouter deux entiers naturels en complément logique ? \\
faire l'addition en base 2; si une retenue déborde (dépasse de la capacité prévue), il faut la réajouter au résultat
\\\hline
Problèmes : addition complexe \\
0 et -0 ont toujours deux représentations différentes \\
Solution : complément arithmétique (ou complément à 2) \\
obtenu en ajoutant 1 au complément logique pour les nombres négatifs\\
Ex: représenter 39 et -39 en base 2, en complément arithmétique sur 8 bits \\
39 = 100111 ; 39 sur 8 bits : 00100111 ;-39 sur 8 bits : 11011000+1 = 11011001 \\
Comment ajouter deux entiers naturels en complément arithmétique ? \\
faire une addition en base 2 \\
si une retenue déborde, il faut l'ignorer
\\ \hline
\end{tabularx} 
\\ \hline
\end{tabular}
\begin{tabularx}{0.5\linewidth}{|Y|}
\hline
\textbf{• 3 Réels}
\\
Pb : il faut un moyen pour représenter les nombres à virgule \\
Solution : format IEEE 754, simple ou double précision \\
signe : bit de signe \\
valeur absolue : mantisse normalisée et exposant (m.$2^{e}$), avec 0.5$\leq$m$<$1 et e entier \\
Mantisse à bit caché \\
le premier bit de la mantisse, qui vaut toujours 1, n'est pas stocké \\
zéro est stocké avec un exposant maximum
\\
Exposant biaisé \\
l'exposant est décalé d'un biais pour que la valeur stockée soit toujours positive \\
Rq : simple précision : 1 bit de signe, 23 bits de mantisse, 8 bits d'exposant (biaisé de -126) \\
double précision : 1 bit de signe, 52 bits de mantisse, 11 bits d'exposant (biaisé de -1022)
\\\hline
Ex : quelle est la valeur décimale de 11000000 00110000 00000000 000000002 ? \\
bit de signe : 1, donc nombre négatif \\
exposant : 10000000=128 (valeur biaisée), 128126=2 (valeur non biaisée) \\
mantisse : 01100...0 (avec bit caché), 101100...0 (sans bit caché) = 1/2+1/8+1/16 = 0,5+0,125+0,0625 = 0,6875\\
résultat : - 0,6875.$2^{2}$ = -2,75
\\\hline
Ex : quel est l'encodage (en simple précision) de 4,25 ? \\
nombre positif, donc bit de signe : 0 \\
4,25 = 2,125*$2^{1}$ = 1,0625*$2^{2}$ = 0,53125*$2^{3}$, arrêt car mantisse normalisée \\
exposant : 3 (valeur non biaisé), 3+126=129 (valeur biaisée), donc : 10000001 \\
mantisse : 0,53125 = 0,5 + 0,03125 = 1/2 + 1/32 = 1000100...0 (sans bit caché), 000100...0 (avec bit caché)\\
résultat : 01000000 10001000 00000000 00000000$_{2}$
\\ \hline
\end{tabularx} 
\begin{tabularx}{0.5\linewidth}{|Y|}
\hline
\textbf{• 4 Caractères}
\\
Comment les caractères sont-ils stockés ? \\
règles associant chaque caractère à un numéro \\
exemple : ASCII, UTF-8, UTF-16, ISO 8859-1 (latin1), etc 
\\ASCII = American Standard Code for Information Interchange \\
table associant chaque caractère (parmi 128) à un numéro sur un octet \\
exemples : 'A' est 65, 'B' est 66, 'a' est 97, '+' est 43
\\
Propriétés \\
les caractères minuscules se suivent \\
les caractères majuscules se suivent \\
les chiffres se suivent \\Problèmes \\
pas de caractères accentués \\
beaucoup (33) de caractères non affichables\\
\\
\textbf{• 5 Chaînes de caractères}
\\
Deux méthodes \\
stocker le nombre de caractères puis la liste des caractères (approche utilisée en Pascal) \\
stocker un caractère terminal (approche utilisée en C ou sous DOS)
\\ \hline
\end{tabularx}
\subsubsection{Circuits logiques}
\begin{tabular}{|c|}
\hline
Circuit logique \\
\begin{tabularx}{\linewidth}{|Y|Y|}
\hline
circuit électronique simplifié permettant de calculer une sortie en fonction d'entrées.
exemple : allumer une lumière en fonction de la position de plusieurs interrupteurs
&
Remarque:
les entrées et les sorties sont composées de signaux binaires.
la manipulation des bits se fait via des fonctions logiques
\\\hline
\end{tabularx}\\
Fonctions logiques :
fonctions de base : NOT, OR, AND ;
autres fonctions : XOR, NOR, NAN\\
\noindent
\begin{tabular}{|c|c|c|c|c|c|}
\hline
Pas A (NOT) & A ou B & A et B & A XOR B &A NOR B & A NAN B \\\hline
\begin{tabular}{|c|c|}
\hline
A=0 & 1 \\\hline
A=1 & 0 \\\hline
\end{tabular} 
&
\begin{tabular}{|c|c|c|}
\hline
& B=0 & B=1\\\hline
A=0 & 0 & 1\\\hline
A=1 & 1 & 1\\\hline
\end{tabular}  
&
\begin{tabular}{|c|c|c|}
\hline
& B=0 & B=1\\\hline
A=0 & 0 & 0\\\hline
A=1 & 0 & 1\\\hline
\end{tabular}
&
\begin{tabular}{|c|c|c|}
\hline
& B=0 & B=1\\\hline
A=0 & 0 & 0\\\hline
A=1 & 0 & 0\\\hline
\end{tabular}
&
\begin{tabular}{|c|c|c|}
\hline
& B=0 & B=1\\\hline
A=0 & 1 & 0\\\hline
A=1 & 0 & 0\\\hline
\end{tabular}
&
\begin{tabular}{|c|c|c|}
\hline
& B=0 & B=1\\\hline
A=0 & 1 & 1\\\hline
A=1 & 1 & 0\\\hline
\end{tabular}\\\hline
\end{tabular} 
\\
\begin{tabularx}{\linewidth}{|Y|Y|}
\hline
Algèbre de Boole \\
$\tab$opérations élémentaires : NOT (noté barre), OR (noté +), AND (noté * ou .) \\
$\tab$permet de représenter des fonctions logiques, de les simplifier, et de faire du calcul (appelé calcul propositionnel) \\
Quelques propriétés
$\tab$•$\overline{A}$=A
$\tab$•lois de De Morgan : $\overline{A+B}$=$\overline{A.B}$ et $\overline{A.B}$=$\overline{A+B}$
\\\hline
Table de vérité \\
$\tab$liste exhaustive des sorties en fonction de toutes les entrées \\
Exemple: la lumière L est allumée uniquement quand les interrupteurs A ou B sont ouverts, à moins que l'interrupteur de sécurité S soit ouvert \\
$\tab$entrées : A, B, S
$\tab$sortie : L
\\\hline
Exemple tableau et fonction
\\\hline
Synthèse \\
$\tab$construction d'un circuit logique à partir d'une fonction logique \\
Étapes de la synthèse:
$\tab$•construction de la table de vérité
$\tab$•obtention d'une expression logique \\
réalisation du circuit
\\\hline
Exemple Circuit schema
\\
Analyse \\
$\tab$construction d'une fonction logique à partir d'un circuit logique \\
Simplification des fonctions logiques (cf TD) \\
$\tab$•par observation de la table de vérité
$\tab$•par propriétés de l'algèbre de Boole
$\tab$•par la méthode des tableaux de Karnaug
\\\hline
\end{tabularx}
\end{tabular} 
\subsubsection{Circuits séquentiels}
\begin{tabularx}{\linewidth}{|Y|Y|}
\hline
Circuit séquentiel \\
•circuit logique qui intègre une notion de temps et de mémoire
•temps : horloge fournissant un signal (à une certaine fréquence)
•mémoire : bascule RS \\\hline
Bascule RS $\tab$•une entrée R (pour reset) : met la sortie à 0
$\tab$•une entrée S (pour set) : met la sortie à 1
\\ \hline
\end{tabularx}
\begin{tabular}{|c|c|}
\hline
Exemple tableau fonction logique :
Qnew=$\overline{R}$.S+$\overline{R}$.Qold=$\overline{R}$.(S+Qold) =$\overline{R}$+$\overline{\overline{(S+Qold)}}$&
Circuit séquentiel de la bascule RS
fonction logique : Q=$\overline{\overline{(S+Q)}}$
\\\hline
\end{tabular}
\\\\
Exemple Circuit schema :
\subsubsection{Architecture d'un ordinateur}
\begin{tabularx}{0.5\linewidth}{|Y|}
\hline
•Ordinateur \\
$\tab$interconnexion de circuits séquentiels + horloge \\
$\tab$bus : moyen de communication entre les circuits séquentiels \\
$\tab$modèle de Von Neumann
\\\hline
\end{tabularx}
\begin{tabularx}{0.5\linewidth}{|Y|}
\hline
•Types de circuits séquentiels \\
$\tab$mémoire \\
$\tab$unités de calculs \\
$\tab$unités de traitement \\
$\tab$périphériques externes
\\\hline
\end{tabularx}
\\
SCHEMA Architecture
\\\\
\textbf{Quelques mots-clés} \\
\noindent
\begin{tabularx}{\linewidth}{|>{\setlength\hsize{0.5\hsize}}Y|>{\setlength\hsize{1.5\hsize}}Y|}
\hline
Expression & Definition \\ \hline
•RAM (Random Access Memory) & mémoire principale, en lecture/écriture 
\\\hline
•ROM (Read-Only Memory) & mémoire en lecture seule 
\\\hline
•BIOS (Basic Input/Output System) & primitives de base de gestion des entrées/sorties, disponibles avant le démarrage du système d'exploitation, sauvegardées en ROM
\\\hline
•CMOS & technologie mémoire historiquement utilisée pour sauvegarder quelques informations du BIOS en mémoire quand l'ordinateur est éteint
\\\hline
•circuit intégré & circuit logique ou séquentiel réalisant une ou plusieurs fonctions 
\\\hline
•processeur & composant électronique réalisant des traitements programmables 
\\\hline
•microprocesseur & processeur réalisé sur un seul circuit intégré
\\ \hline
\end{tabularx} 
 \\
\subsection{Résumé des principales connaissances à avoir}\begin{tabular}{|l|}
\hline
Résumé des principales connaissances à avoir \\
Codage : entiers positifs, entiers naturels, réels, caractères \\
Circuits logiques : synthèse, analyse \\
Circuits séquentiels : principe \\
Ordinateur : organisation
\\ \hline
\end{tabular}
\subsection{titre}
\end{scriptsize}
\end{document}