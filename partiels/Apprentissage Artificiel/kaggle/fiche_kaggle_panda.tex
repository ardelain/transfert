\documentclass{article}
\usepackage{array, tabularx}
\newcolumntype{Y}{>{\centering\arraybackslash}X}

\newdimen\demilargeur
\demilargeur=\textwidth
\divide\demilargeur by 2

\title{UTILISATION KAGGLE}
\date{}
\begin{document}
\maketitle
    \part*{Importation et creation de données}
    
	\section*{Creation de donnees}
	Les dataframes : \\
	\texttt{$donees = pd.DataFrame(\{ 
	"column" : ["donnee1","donnee2"],  
	"column2" : ["donnee2-1","donnee2-2"],... \}
	$}  
	\\ \\Les series : \\
    \texttt{$donnees_t = pd.Series(
    ["donnee1","donnee2","donnee1"],
    index=["index1","index2","index3"],
    name="nomSerie")$s}
	
    
    \section*{Importation de donnees}
    CSV:
    \\
     \texttt{$GET : pd.read\_csv("../input/fichier/nom.csv",index\_col=0)$}\\
    \texttt{$PUT : pd.DataFrame.to\_csv(path\_or\_buf='./nom.csv',self=nom)$}
    \\ \\SQL:\\
    
    \section*{Manipulation des structures de donnees}
    
    \begin{tabular}{|p{0.33\linewidth}|p{0.33\linewidth}|p{0.33\linewidth}|}
	\hline
	\multicolumn{1}{|c|}{\textbf{Code}} &
	\multicolumn{1}{c|}{\textbf{Description}} &
	\multicolumn{1}{c|}{\textbf{Exemple}} \\ \hline
	.rename(columns= \texttt{$\{ 'nom_column':'nv_nom',...\}$} & changer le nom des colonnes & ... \\ \hline
	.dropna() & remove all the rows that contain a missing value & ...  \\  \hline
	.arange(n) & ... & ...  \\ \hline
	resize((n,m)) & ... & ...  \\ \hline
	.copy() & ... & copy original   \\  \hline
	.reset\_index() & ... & copy original   \\  
	\hline
	\end{tabular}
    \section*{Information sur les structures de donnees}
    \begin{tabular}{|p{0.33\linewidth}|p{0.33\linewidth}|p{0.33\linewidth}|}
	\hline
	\multicolumn{1}{|l|}{Code} &
	\multicolumn{1}{c|}{Description} &
	\multicolumn{1}{c|}{Exemple} \\ \hline
	.shape & nombre de ligne et de colonnes & ... \\ \hline
	.columns & ... & ...  \\ \hline
	.info() & information generale sur la dataframe. Type de donnée : bool, int64, float64 and object & ...  \\ \hline
	.describe() & multiple info : 
nombre de valeurs non manquantes, moyenne, ecart type, etendue, médiane, quartiles et 0,25 & ...  \\ \hline
	.describe(include= ['object', 'bool']) & statistiques sur les caractéristique non numerique ( indiquer explicitement les types de donnees d'intérêt dans le parametre include ) & ...  \\ \hline  
	... & ... & ... \\ 
	\hline
	\end{tabular}
	
    \part*{Selection et manipulation des donnees}
    
    \section*{Selection des donnees}
	\begin{tabular}{|l|c|c|}
	\hline
	\multicolumn{1}{|l|}{Code} &
	\multicolumn{1}{c|}{Description} &
	\multicolumn{1}{c|}{Exemple} \\ \hline
	donnees.nom\_column & Data frame de la colonne en question & ... \\ \hline
	.nom\_column.iloc[n] & n ieme valeur de la colonne nom\_column  & ...  \\ \hline
	donnees.iloc[i,j] & valeur colonne i et ligne j(exclu)  & ... \\ \hline
	(n).iloc[i,j] & [:n] (tout jusqu'a n) et iloc & ...  \\ \hline
	.groupby() & grouper les donnees selon un critere nom & ...  \\ \hline
	.sample(donnees) & ... & ... \\ \hline
	\hline
	\end{tabular}
	\\ \\\\
    Condition:\\
    \texttt{$==,<,>,<=,>=$} comparaison de valeur avec donnee selectionnee
	\\ \\
	Reference logique
	\\
    \~\ pour le not;	\texttt{$|$} pour le or;	\& pour le and\\

    \section*{Informations sur les donnees}

	Donnee seul : \\
    \noindent
    \begin{tabular}{|p{0.33\linewidth}|p{0.33\linewidth}|p{0.33\linewidth}|}
	\hline
	\multicolumn{1}{|l|}{Code} &
	\multicolumn{1}{c|}{Description} &
	\multicolumn{1}{c|}{Exemple} \\ \hline
	.dtype & ... & ... \\ \hline
	.astype('type') & convertion de la donee & ... \\ \hline
	.isnull() & ... & ... \\ \hline
	.fillna('nom') & ... & ... \\ \hline		
	... & ... & ...  \\ 
	\hline
	\end{tabular}
	

	\noindent
	\\Ensemble de Donnee : \\
	\noindent
    \begin{tabular}{|p{0.33\linewidth}|p{0.33\linewidth}|p{0.33\linewidth}|}
	\hline
	\multicolumn{1}{|l|}{Code} &
	\multicolumn{1}{c|}{Description} &
	\multicolumn{1}{c|}{Exemple} \\ \hline
	.count() & ... & ... \\ \hline
	.sum() & ... & ... \\ \hline
	.max() & donnee max & ... \\ \hline
	.min() & donnee min & ... \\ \hline
	.mean() & moyenne & ... \\ \hline
	.std() & ... & ... \\ \hline
	.argmax() & index donnee max & ... \\ \hline
	.argmin() & index donnee min & ... \\ \hline
	stat.mods & ... & ... \\ \hline
	... & ... & ... \\ \hline
	... & ... & ... \\ 
	\hline
	\end{tabular}
	
	 \section*{Manipulation des donnees}
  
	\begin{center}
    \begin{tabular}{|p{0.33\linewidth}|p{0.33\linewidth}|p{0.33\linewidth}|}
	\hline
	\multicolumn{1}{|l|}{Code} &
	\multicolumn{1}{c|}{Description} &
	\multicolumn{1}{c|}{Exemple} \\ \hline
	data.unique() & supprime les doublons & ...  \\
	\hline
	\end{tabular}
    \end{center}
    
    \section*{Autre}
    
    \begin{center}
    \begin{tabular}{|p{0.33\linewidth}|p{0.33\linewidth}|p{0.33\linewidth}|}
	\hline
	\multicolumn{1}{|l|}{Code} &
	\multicolumn{1}{c|}{Description} &
	\multicolumn{1}{c|}{Exemple} \\ \hline
	data. & ... & ...  \\
	\hline
	\end{tabular}
    \end{center}
    
    \part*{Graphiques}
    
     \section*{Simples}
     \section*{Complexes}
     \section*{Autres}
\end{document}